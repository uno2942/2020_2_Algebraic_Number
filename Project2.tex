%Calculus Homework
\documentclass[a4paper, 12pt]{article}

%================================================================================
%Package
	\usepackage{amsmath, amsthm, amssymb, latexsym, mathtools, mathrsfs, physics}
	\usepackage{dsfont, txfonts, soul, stackrel, tikz-cd, graphicx, titlesec, etoolbox}
	\DeclareGraphicsExtensions{.pdf,.png,.jpg}
	\usepackage{fancyhdr}
	\usepackage[shortlabels]{enumitem}
	\usepackage[pdfmenubar=true, pdfborder	={0 0 0 [3 3]}]{hyperref}
	\usepackage{kotex}

%================================================================================
\usepackage{verbatim}
\usepackage{physics}
\usepackage{makebox}
\usepackage{pst-node}

%================================================================================
%Layout
	%Page layout
	\addtolength{\hoffset}{-50pt}
	\addtolength{\headheight}{+10pt}
	\addtolength{\textwidth}{+75pt}
	\addtolength{\voffset}{-50pt}
	\addtolength{\textheight}{+75pt}
	\newcommand{\Space}{1em}
	\newcommand{\Vspace}{\vspace{\Space}}
	\newcommand{\ran}{\textrm{ran }}
	\setenumerate{listparindent=\parindent}

%================================================================================
%Statement
	\newtheoremstyle{Mytheorem}%
	{1em}{1em}%
	{\slshape}{}%
	{\bfseries}{.}%
	{ }{}

	\newtheoremstyle{Mydefinition}%
	{1em}{1em}%
	{}{}%
	{\bfseries}{.}%
	{ }{}

	\theoremstyle{Mydefinition}
	\newtheorem{statement}{Statement}
	\newtheorem{definition}[statement]{Definition}
	\newtheorem{definitions}[statement]{Definitions}
	\newtheorem{remark}[statement]{Remark}
	\newtheorem{remarks}[statement]{Remarks}
	\newtheorem{example}[statement]{Example}
	\newtheorem{examples}[statement]{Examples}
	\newtheorem{question}[statement]{Question}
	\newtheorem{questions}[statement]{Questions}
	\newtheorem{problem}[statement]{Problem}
	\newtheorem{exercise}{Exercise}[section]
	\newtheorem*{comment*}{Comment}
	%\newtheorem{exercise}{Exercise}[subsection]

	\theoremstyle{Mytheorem}
	\newtheorem{theorem}[statement]{Theorem}
	\newtheorem{corollary}[statement]{Corollary}
	\newtheorem{corollaries}[statement]{Corollaries}
	\newtheorem{proposition}[statement]{Proposition}
	\newtheorem{lemma}[statement]{Lemma}
	\newtheorem{claim}{Claim}
	\newtheorem{claimproof}{Proof of claim}[claim]
	\newenvironment{myproof1}[1][\proofname]{%
  \proof[\textit Proof of problem #1]%
}{\endproof}

%================================================================================
%Header & footer
	\fancypagestyle{myfency}{%Plain
	\fancyhf{}
	\fancyhead[L]{}
	\fancyhead[C]{}
	\fancyhead[R]{}
	\fancyfoot[L]{}
	\fancyfoot[C]{}
	\fancyfoot[R]{\thepage}
	\renewcommand{\headrulewidth}{0.4pt}
	\renewcommand{\footrulewidth}{0pt}}

	\fancypagestyle{myfirstpage}{%Firstpage
	\fancyhf{}
	\fancyhead[L]{}
	\fancyhead[C]{}
	\fancyhead[R]{}
	\fancyfoot[L]{}
	\fancyfoot[C]{}
	\fancyfoot[R]{\thepage}
	\renewcommand{\headrulewidth}{0pt}
	\renewcommand{\footrulewidth}{0pt}}

	\pagestyle{myfency}

%================================================================================

%***************************
%*** Additional Command ****
%***************************

\DeclareMathOperator{\cl}{cl}
\DeclareMathOperator{\co}{co}
\DeclareMathOperator{\ball}{ball}
\DeclareMathOperator{\wk}{wk}
\DeclarePairedDelimiter{\ceil}{\lceil}{\rceil}
\DeclarePairedDelimiter\floor{\lfloor}{\rfloor}
\newcommand{\quotZ}[1]{\ensuremath{\mathbb{Z}/p^{#1}\mathbb{Z}}}
%================================================================================
%Document
\begin{document}
\thispagestyle{myfirstpage}
\begin{center}
	\Large{HW9}
\end{center}
박성빈, 수학과

\noindent \textbf{1} Any natural number $n\geq 2$ has unique prime factorization. Let $n=\prod_{i} p_i^{k_i}$ for distinct primes $\{p_i\}$ and $k_i\geq 1$, then
\begin{equation}
    \sigma_k(n) = \prod_i\left(\sum_{j=0}^{k_i} p_i^{kj}\right) = \prod_i\left(\frac{p_i^{k(k_i+1)}-1}{p_i^k-1}\right)
\end{equation}
In this setting, we get the decomposition by
\begin{equation}
\begin{split}
    \sum_{n\geq 1}\frac{\sigma_k(n)}{n^s} &= \prod_{p_i:\textrm{primes}}\left(\sum_{k_i=0}^\infty \frac{1}{p_i^{k_is}}\left(\frac{p_i^{k(k_i+1)}-1}{p_i^k-1}\right)\right)\\
    &=\left(1+\left(\frac{2^{2k}-1}{2^k-1}\right)\frac{1}{2^s} + \left(\frac{2^{3k}-1}{2^k-1}\right)\frac{1}{4^s} + \cdots\right)\left(1+\left(\frac{3^{2k}-1}{3^k-1}\right)\frac{1}{3^s} + \left(\frac{3^{3k}-1}{3^k-1}\right)\frac{1}{9^s} + \cdots\right)\cdots
\end{split}
\end{equation}
I'll show that the both side of above equation absolutely converges if $\Re(s)>k+1$ and indeed equal. For LHS with fixed $n=\prod_{i} p_i^{k_i}$,
\begin{equation}
    \sigma_k(n) = \prod_i\left(\frac{p_i^{k(k_i+1)}-1}{p_i^k-1}\right)\leq n^k\prod_i \frac{p_i^k}{p_i^k-1} = n^k\prod_i \left(1+\frac{1}{p_i^k-1}\right) \leq n^k \exp\left(\sum_{i}\frac{1}{p_i^k-1}\right),
\end{equation}
and
\begin{equation}
    \exp\left(\sum_{i}\frac{1}{p_i^k-1}\right)\leq \exp\left(1+\sum_{i}\frac{1}{p_i}\right)
\end{equation}
since $p_i$ are monotonically increasing and considering $p_i=2$ case. Using the fact that 
\begin{equation}
    \lim_{n\rightarrow \infty}\left(\sum_{p_i\leq n}\frac{1}{p_i}-\log\log n\right) \leq C
\end{equation}
for some constant $C$, which is in fact Meissel-Mertens constant, there exists a constant $M$ such that
\begin{equation}
    \exp\left(1+\sum_{i}\frac{1}{p_i}\right)\leq \exp\left(M+\log\log n\right) = e^M \log n.
\end{equation}
Therefore,
\begin{equation}
    \sum_{n\geq 1}\abs{\frac{\sigma_k(n)}{n^s}}\leq 1+ \sum_{n\geq 2}\frac{e^M \log n}{n^{\Re(s)-k}}<\infty
\end{equation}
if $\Re(s)>k+1$.
Also, the RHS converges: for fixed $p_i$ and $k$,
\begin{equation}
    \begin{split}
    \sum_{k_i=1}^\infty \frac{1}{p_i^{k_i\Re(s)}}\left(\frac{p_i^{k(k_i+1)}-1}{p_i^k-1}\right) &\leq \left(\frac{p_i^k}{p_i^k-1}\right)\sum_{k_i=1}^\infty \frac{1}{p_i^{k_i\Re(s)}}p_i^{kk_i}\\
    &=\left(\frac{p_i^k}{p_i^k-1}\right)\sum_{k_i=1}^\infty p_i^{(k-\Re(s))k_i}\\
    &=\left(\frac{p_i^k}{p_i^k-1}\right)\frac{p_i^{k-\Re(s)}}{1-p_i^{k-\Re(s)}}\leq  \frac{4}{p_i^{\Re(s)-k}},
    \end{split}
\end{equation}
and
\begin{equation}
    \sum_{p_i:\textrm{primes}}\frac{4}{p_i^{\Re(s)-k}}\leq \sum_{n=1}^\infty \frac{4}{n^{\Re(s)-k}}<\infty,
\end{equation}
so the infinite product absolutely converges. By comparing elements in both side, we get the equality.

Now, I'll show that
\begin{equation}
    \zeta(s)\zeta(s-k) = \sum_{n\geq 1}\frac{\sigma_k(n)}{n^s}.
\end{equation}
for above setting. Since $\Re(s)>k+1$, the series form of $\zeta(s)$ and $\zeta(s-k)$ absolutely converges, so we can interchange the position of each term, and we get
\begin{equation}
\begin{split}
    \zeta(s)\zeta(s-k) &= \prod_{p_i\geq 2}\left(1+\frac{1}{p_i^s}+\frac{1}{p_i^{2s}}+\cdots\right)\left(1+\frac{1}{p_i^{s-k}}+\frac{1}{p_i^{2(s-k)}}+\cdots\right)\\
    &=\prod_{p_i\geq 2}\left(1+\frac{1}{p_i^s}+\frac{1}{p_i^{2s}}+\cdots\right)\left(1+\frac{p_i^k}{p_i^{s}}+\frac{p_i^{2k}}{p_i^{2s}}+\cdots\right)\\
    &=\prod_{p_i\geq 2}\left(\sum_{n=0}^\infty \frac{\sum_{j=0}^n p_i^{kj}}{p_i^{ns}}\right)\\
    &=\sum_{n\geq 1}\frac{\sigma_k(n)}{n^s}.
\end{split}
\end{equation}

\noindent \textbf{2}
Using Iwasawa decomposition, we write
\begin{equation}
    g = \gamma \cdot \left\{\begin{pmatrix}
    y_\infty & x_\infty \\
    0 & 1
    \end{pmatrix}\begin{pmatrix}
    r_\infty & 0 \\
    0 & r_\infty
    \end{pmatrix}, I_2, \ldots, \right\}\cdot k
\end{equation}
where $\gamma=\begin{pmatrix}
    \gamma_{1,1} & \gamma_{1,2}\\
    \gamma_{2,1} & \gamma_{2,2}
    \end{pmatrix}\in \mathrm{GL}_2\mathbb{Q}$, $-1/2\leq x_\infty\leq 0$, $y_\infty>0$, $x_\infty^2+y_\infty^2\geq 1$, $r_\infty>0$, $I_2 = \begin{pmatrix}
1 & 0\\
0 & 1
\end{pmatrix}$, and $k\in \mathrm{O}_2\mathbb{R}\cdot \prod_p \mathrm{GL}_2\mathbb{Z}_p$.
I'll divide $\infty$ case and finite case and compute the lower bound of $\norm{g}$.

For $\infty$ case, we get
\begin{equation}
    \begin{pmatrix}
    \gamma_{1,1} & \gamma_{1,2}\\
    \gamma_{2,1} & \gamma_{2,2}
    \end{pmatrix} \begin{pmatrix}y_\infty & x_\infty\\ 0 & 1
    \end{pmatrix}\begin{pmatrix}r_\infty & 0\\ 0 & r_\infty
    \end{pmatrix} = \begin{pmatrix}
    \gamma_{1,1} & \gamma_{1,2}\\
    \gamma_{2,1} & \gamma_{2,2}
    \end{pmatrix} \begin{pmatrix}r_\infty y_\infty & r_\infty x_\infty\\ 0 & r_\infty
    \end{pmatrix} = \begin{pmatrix}
    r_\infty y_\infty\gamma_{1,1} & r_\infty(x_\infty\gamma_{1,1} + \gamma_{1,2})\\
    r_\infty y_\infty\gamma_{2,1} & r_\infty(x_\infty\gamma_{2,1} + \gamma_{2,2})
    \end{pmatrix},
\end{equation}
so
\begin{equation}
    \norm{\begin{pmatrix}
    \gamma_{1,1} & \gamma_{1,2}\\
    \gamma_{2,1} & \gamma_{2,2}
    \end{pmatrix} \begin{pmatrix}y_\infty & x_\infty\\ 0 & 1
    \end{pmatrix}\begin{pmatrix}r_\infty & 0\\ 0 & r_\infty
    \end{pmatrix}}_\infty \geq \max\left(\abs{\gamma_{1,1}}_\infty y_\infty r_\infty, \abs{\det(\gamma)}_\infty^{-1}y_\infty^{-1} r_\infty^{-2}\right).
\end{equation}
For $h_\infty = \begin{pmatrix}a_\infty & b_\infty \\ c_\infty & d_\infty\end{pmatrix}\in \mathrm{GL}_2\mathbb{R}$ and $k_\infty\in \mathrm{O}_2\mathbb{R}$,
\begin{equation}
    h_\infty k_\infty = \begin{pmatrix}
    a_\infty & b_\infty\\
    c_\infty & d_\infty
    \end{pmatrix}\begin{pmatrix}
    \cos \theta_\infty & \mp\sin\theta_\infty\\
    \sin \theta_\infty & \pm\cos \theta_\infty
    \end{pmatrix}
    =\begin{pmatrix}
    a_\infty \cos\theta_\infty + b_\infty\sin\theta_\infty & \mp a_\infty\sin\theta_\infty \pm b_\infty \cos\theta_\infty \\
    c_\infty \cos\theta_\infty + d_\infty\sin\theta_\infty & \mp c_\infty\sin\theta_\infty \pm d_\infty \cos\theta_\infty
    \end{pmatrix}.
\end{equation}
By denoting $s = a_\infty \cos\theta_\infty + b_\infty\sin\theta_\infty$ and $t = \mp a_\infty\sin\theta_\infty \pm b_\infty \cos\theta_\infty$, we get $s^2+t^2 = a_\infty^2+b_\infty^2$, and making $\max(\abs{s},\abs{t})$ minimum is the point $\abs{s}=\abs{t}=\sqrt{\frac{a_\infty^2+b_\infty^2}{2}}$. Note that $\sqrt{a_\infty^2+b_\infty^2}\geq \abs{a_\infty}_\infty, \abs{b_\infty}_\infty$. Same argument applies for $c_\infty$ and $d_\infty$ places, so we get
\begin{equation}
    \norm{h_\infty k_\infty}_\infty \geq \frac{1}{\sqrt{2}}\norm{h_\infty}_\infty.
\end{equation}
as multiplying $k_\infty$ does not change the determinant. It shows
\begin{equation}
    \norm{\gamma\cdot\begin{pmatrix}y_\infty & x_\infty\\ 0 & 1
    \end{pmatrix}\begin{pmatrix}r_\infty & 0\\ 0 & r_\infty
    \end{pmatrix}\cdot k}_\infty \geq \frac{1}{\sqrt{2}}\max\left(\abs{\gamma_{1,1}}_\infty y_\infty r_\infty, \abs{\det(\gamma)}_\infty^{-1}y_\infty^{-1} r_\infty^{-2}\right).
\end{equation}

Now, let's consider finite prime $p$ case. Since $k_p=\begin{pmatrix}
p_{1,1} & p_{1,2}\\
p_{2,1} & p_{2,2}
\end{pmatrix}\in \mathrm{GL}_2\mathbb{Z}_p$, $\max(\abs{p_{1,1}p_{2,2}}_p,\abs{p_{1,2}p_{2,1}}_p) = 1$, which means that $p_{1,1},p_{2,2} \in\mathbb{Z}_p^\times$ or $p_{1,2},p_{2,1}\in\mathbb{Z}_p^\times$. Assume $p_{1,1},p_{2,2} \in\mathbb{Z}_p^\times$. Computing $\gamma\cdot k_p$, we get
\begin{equation}
    \gamma\cdot k_p = \begin{pmatrix}
    \gamma_{1,1} & \gamma_{1,2}\\
    \gamma_{2,1} & \gamma_{2,2}
    \end{pmatrix}\begin{pmatrix}
    p_{1,1} & p_{1,2}\\
    p_{2,1} & \gamma_{2,2}
    \end{pmatrix} = \begin{pmatrix}
    \gamma_{1,1}p_{1,1} + \gamma_{1,2}p_{2,1} & \gamma_{1,1}p_{1,2} + \gamma_{1,2}p_{2,2}\\
    \gamma_{2,1}p_{1,1} + \gamma_{2,2}p_{2,1} & \gamma_{2,1}p_{1,2} + \gamma_{2,2}p_{2,2}
    \end{pmatrix}.
\end{equation}
If $\abs{\gamma_{1,1}}_p\geq\abs{\gamma_{1,2}}_p$, then $\abs{\gamma_{1,1}}_p = \abs{\gamma_{1,1}p_{1,1}}_p \geq \abs{\gamma_{1,2}p_{2,1}}_p$ and 
\begin{equation}
    \abs{\gamma_{1,1}p_{1,2} + \gamma_{1,2}p_{2,2}}_p\leq \max\left(\abs{\gamma_{1,1}}_p \abs{p_{1,2}}_p, \abs{\gamma_{1,2}}_p\right)\leq \abs{\gamma_{1,1}}_p.
\end{equation} 
In this case
\begin{equation}
\begin{split}
    \max\left(\abs{\gamma_{1,1}p_{1,1} + \gamma_{1,2}p_{2,1}}_p, \abs{\gamma_{1,1}p_{1,2} + \gamma_{1,2}p_{2,2}}_p\right) &= \abs{\gamma_{1,1}}_p
\end{split}
\end{equation}
Conversely, if $\abs{\gamma_{1,1}}_p<\abs{\gamma_{1,2}}_p$, by the same reason, we get
\begin{equation}
    \max\left(\abs{\gamma_{1,1}p_{1,1} + \gamma_{1,2}p_{2,1}}_p, \abs{\gamma_{1,1}p_{1,2} + \gamma_{1,2}p_{2,2}}_p\right) = \abs{\gamma_{1,2}}_p
\end{equation}
Applying this result to $\gamma_{2,1}$ and $\gamma_{2,2}$ position, we finally get
\begin{equation}
    \norm{\gamma\cdot k_p}_p = \norm{\gamma}_p \geq \max\left(\abs{\gamma_{1,1}}_p, \abs{\det{\gamma}}_p^{-1}\right)
\end{equation}
as $\norm{k_p}_p=1$, so does not change the determinant part. The same computation goes for the case $p_{1,2},p_{2,1}\in\mathbb{Z}_p^\times$. Therefore, the above result holds for any $k_p\in\mathrm{GL}_2\mathbb{Z}_p$.

Composing the results of infinite and finite cases,
\begin{equation}
\begin{split}
    \norm{g} &= \prod_v \norm{g}_v\geq \frac{1}{\sqrt{2}}\max\left(\abs{\gamma_{1,1}}_\infty y_\infty r_\infty, \abs{\det(\gamma)}_\infty^{-1}y_\infty^{-1} r_\infty^{-2}\right)\prod_{p}\max\left(\abs{\gamma_{1,1}}_p, \abs{\det{\gamma}}_p^{-1}\right)\\
    &\geq \frac{1}{\sqrt{2}}\max\left(\left(\prod_v\abs{\gamma_{1,1}}_v\right) y_\infty r_\infty, \left(\prod_v \abs{\det{\gamma}}_v^{-1}\right) y_\infty^{-1} r_\infty^{-2}\right)\\
    &=\frac{1}{\sqrt{2}}\max\left(y_\infty r_\infty, y_\infty^{-1} r_\infty^{-2}\right)\\
    &\geq \frac{1}{\sqrt{2}}y_\infty^{1/3}
\end{split}
\end{equation}
where $p$ represents finite primes, and $v$ represents primes including $\infty$. From second to third line, I used the fact that for $q\in \mathbb{Q}$, $\prod_v\abs{q}_v = 1$. From third to fourth line, I used that minimizing the $\max\left(y_\infty r_\infty, y_\infty^{-1} r_\infty^{-2}\right)$ is the case $y_\infty r_\infty = y_\infty^{-1}r_\infty^{-2}$ since writing $t=y_\infty r_\infty$ and $s = y_\infty^{-1}r_\infty^{-2}$, $t^2s = y_\infty$, so the minimizing point is the intersection of square at $0$ in $s-t$ plane and $s=y_\infty/t^2$.
%________________________________________________________________________
\end{document}

%================================================================================
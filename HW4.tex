%Calculus Homework
\documentclass[a4paper, 12pt]{article}

%================================================================================
%Package
	\usepackage{amsmath, amsthm, amssymb, latexsym, mathtools, mathrsfs, physics}
	\usepackage{dsfont, txfonts, soul, stackrel, tikz-cd, graphicx, titlesec, etoolbox}
	\DeclareGraphicsExtensions{.pdf,.png,.jpg}
	\usepackage{fancyhdr}
	\usepackage[shortlabels]{enumitem}
	\usepackage[pdfmenubar=true, pdfborder	={0 0 0 [3 3]}]{hyperref}
	\usepackage{kotex}

%================================================================================
\usepackage{verbatim}
\usepackage{physics}
\usepackage{makebox}
\usepackage{pst-node}

%================================================================================
%Layout
	%Page layout
	\addtolength{\hoffset}{-50pt}
	\addtolength{\headheight}{+10pt}
	\addtolength{\textwidth}{+75pt}
	\addtolength{\voffset}{-50pt}
	\addtolength{\textheight}{+75pt}
	\newcommand{\Space}{1em}
	\newcommand{\Vspace}{\vspace{\Space}}
	\newcommand{\ran}{\textrm{ran }}
	\setenumerate{listparindent=\parindent}

%================================================================================
%Statement
	\newtheoremstyle{Mytheorem}%
	{1em}{1em}%
	{\slshape}{}%
	{\bfseries}{.}%
	{ }{}

	\newtheoremstyle{Mydefinition}%
	{1em}{1em}%
	{}{}%
	{\bfseries}{.}%
	{ }{}

	\theoremstyle{Mydefinition}
	\newtheorem{statement}{Statement}
	\newtheorem{definition}[statement]{Definition}
	\newtheorem{definitions}[statement]{Definitions}
	\newtheorem{remark}[statement]{Remark}
	\newtheorem{remarks}[statement]{Remarks}
	\newtheorem{example}[statement]{Example}
	\newtheorem{examples}[statement]{Examples}
	\newtheorem{question}[statement]{Question}
	\newtheorem{questions}[statement]{Questions}
	\newtheorem{problem}[statement]{Problem}
	\newtheorem{exercise}{Exercise}[section]
	\newtheorem*{comment*}{Comment}
	%\newtheorem{exercise}{Exercise}[subsection]

	\theoremstyle{Mytheorem}
	\newtheorem{theorem}[statement]{Theorem}
	\newtheorem{corollary}[statement]{Corollary}
	\newtheorem{corollaries}[statement]{Corollaries}
	\newtheorem{proposition}[statement]{Proposition}
	\newtheorem{lemma}[statement]{Lemma}
	\newtheorem{claim}{Claim}
	\newtheorem{claimproof}{Proof of claim}[claim]
	\newenvironment{myproof1}[1][\proofname]{%
  \proof[\textit Proof of problem #1]%
}{\endproof}

%================================================================================
%Header & footer
	\fancypagestyle{myfency}{%Plain
	\fancyhf{}
	\fancyhead[L]{}
	\fancyhead[C]{}
	\fancyhead[R]{}
	\fancyfoot[L]{}
	\fancyfoot[C]{}
	\fancyfoot[R]{\thepage}
	\renewcommand{\headrulewidth}{0.4pt}
	\renewcommand{\footrulewidth}{0pt}}

	\fancypagestyle{myfirstpage}{%Firstpage
	\fancyhf{}
	\fancyhead[L]{}
	\fancyhead[C]{}
	\fancyhead[R]{}
	\fancyfoot[L]{}
	\fancyfoot[C]{}
	\fancyfoot[R]{\thepage}
	\renewcommand{\headrulewidth}{0pt}
	\renewcommand{\footrulewidth}{0pt}}

	\pagestyle{myfency}

%================================================================================

%***************************
%*** Additional Command ****
%***************************

\DeclareMathOperator{\cl}{cl}
\DeclareMathOperator{\co}{co}
\DeclareMathOperator{\ball}{ball}
\DeclareMathOperator{\wk}{wk}
\DeclarePairedDelimiter{\ceil}{\lceil}{\rceil}
\DeclarePairedDelimiter\floor{\lfloor}{\rfloor}
\newcommand{\quotZ}[1]{\ensuremath{\mathbb{Z}/p^{#1}\mathbb{Z}}}
%================================================================================
%Document
\begin{document}
\thispagestyle{myfirstpage}
\begin{center}
	\Large{HW4}
\end{center}
박성빈, 수학과, 20202120

\noindent \textbf{1}
Let $U_n = (1/n, \infty)$. Then $\cup_{n=1}^\infty U_n = \mathbb{R}_+$. Since any compact set $C$ is covered by $\{U_n\}$, there exists $N\in\mathbb{N}$ such that $C\subset U_N$. Using L'H\^opital's theorem, we get $\lim_{t\rightarrow \infty} te^{-t} = 0$, so there exists $M\in\mathbb{N}$ such that for $t\geq M$, $e^{-t}\leq 1/t$.

For any $t\in C$, $t\geq 1/N$ and
\begin{equation}
    \theta(t) = \sum_{n^2<\frac{M}{\pi t}} e^{-\pi n^2 t} + \sum_{n^2\geq \frac{M}{\pi t}} e^{-\pi n^2 t}\leq \sum_{n^2<\frac{M}{\pi t}} e^{-\pi n^2 t} + \sum_{n^2\geq \frac{M}{\pi t}} \frac{1}{\pi tn^2}<\infty.
\end{equation}
Therefore, it converges absolutely. For any $\epsilon>0$, there exists $L\in\mathbb{N}$ such that $\sum_{n=L}^\infty \frac{1}{n^2}<\frac{\pi}{3N}\epsilon$, so if we set $M' = \max\{\frac{M}{\pi N}, L^2\}$, then
\begin{equation}
    \abs{\theta(t)-\sum_{n^2<M'} e^{-\pi n^2 t}}\leq \sum_{n^2\geq M'} \frac{1}{\pi tn^2}<\frac{2\epsilon}{3}.
\end{equation}
for all $t\in C\subset U_N$. Therefore, $\theta(t)$ converges uniformly on $C$.\\

\noindent \textbf{2}
Let
\begin{equation}
    \hat{f}(\xi) = \int_{-\infty}^\infty e^{-\pi x^2} e^{-2\pi i \xi x }dx.
\end{equation}

Since $e^{-\pi x^2}$ is in Schwartz space, $\hat{f}$ is also in Schwartz space and
\begin{equation}
    \left(\hat{f}\right)'(\xi) =  \int_{-\infty}^\infty e^{-\pi x^2} \pdv{\xi}\left(e^{-2\pi i x \xi}\right)dx = -2\pi i \widehat{\left(xf\right)}(\xi).
\end{equation}
Also, computing $\widehat{\left(f'\right)}(\xi)$, we get
\begin{equation}
\begin{split}
    \widehat{\left(f'\right)}(\xi) &= \int_{-\infty}^\infty \pdv{x}\left(e^{-\pi x^2}\right)e^{-2\pi i x \xi} dx = -\int_{-\infty}^\infty e^{-\pi x^2}\pdv{x}\left(e^{-2\pi i x \xi}\right) dx = 2\pi i \xi\hat{f}(\xi)\\
    &=-2\pi\widehat{\left(xf\right)}(\xi)
\end{split}
\end{equation}

Therefore, $\hat{f}$ follows the ODE
\begin{equation}
    \left(\hat{f}\right)'(\xi) = -2\pi\xi\hat{f}(\xi)
\end{equation}
with $\hat{f}(0) = 1$. Solving the differential equation, we get $\hat{f}(\xi) = e^{-\pi \xi^2}$.\\

\noindent \textbf{3}

Fix $t\neq 0$. Set $g_t(x) = e^{-\pi t x^2}$. For $t\neq 0$, $g$ is again in Schwartz space. Using Poisson summation formula,
\begin{equation}
    \theta(t) = \sum_{n\in \mathbb{Z}} g_t(n) = \sum_{n\in \mathbb{Z}} \hat{g_t}(n)
\end{equation}
and
\begin{equation}
    \hat{g_t}(\xi) = \int_{-\infty}^\infty e^{-\pi t x^2} e^{-2\pi i \xi x}dx = \frac{1}{\sqrt{t}}\int_{-\infty}^\infty \exp{-\pi x^2 -\frac{2\pi i \xi x}{\sqrt{t}}}dx = \frac{1}{\sqrt{t}}\hat{f}\left(\frac{\xi}{\sqrt{t}}\right) = \frac{1}{\sqrt{t}}g_t\left(\frac{\xi}{t}\right) = \frac{1}{\sqrt{t}}g_{t^{-1}}\left(\xi\right),
\end{equation}
where $f$ is defined in problem 2.

Therefore, we get
\begin{equation}
    \theta(t) = \sum_{n\in \mathbb{Z}} \hat{g_t}(n) = \frac{1}{\sqrt{t}}\sum_{n\in \mathbb{Z}} g_{t^{-1}}\left(n\right) = \frac{1}{\sqrt{t}}\theta\left(\frac{1}{t}\right).
\end{equation}
%________________________________________________________________________
\end{document}

%================================================================================
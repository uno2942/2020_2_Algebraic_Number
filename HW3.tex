%Calculus Homework
\documentclass[a4paper, 12pt]{article}

%================================================================================
%Package
	\usepackage{amsmath, amsthm, amssymb, latexsym, mathtools, mathrsfs, physics}
	\usepackage{dsfont, txfonts, soul, stackrel, tikz-cd, graphicx, titlesec, etoolbox}
	\DeclareGraphicsExtensions{.pdf,.png,.jpg}
	\usepackage{fancyhdr}
	\usepackage[shortlabels]{enumitem}
	\usepackage[pdfmenubar=true, pdfborder	={0 0 0 [3 3]}]{hyperref}
	\usepackage{kotex}

%================================================================================
\usepackage{verbatim}
\usepackage{physics}
\usepackage{makebox}
\usepackage{pst-node}

%================================================================================
%Layout
	%Page layout
	\addtolength{\hoffset}{-50pt}
	\addtolength{\headheight}{+10pt}
	\addtolength{\textwidth}{+75pt}
	\addtolength{\voffset}{-50pt}
	\addtolength{\textheight}{+75pt}
	\newcommand{\Space}{1em}
	\newcommand{\Vspace}{\vspace{\Space}}
	\newcommand{\ran}{\textrm{ran }}
	\setenumerate{listparindent=\parindent}

%================================================================================
%Statement
	\newtheoremstyle{Mytheorem}%
	{1em}{1em}%
	{\slshape}{}%
	{\bfseries}{.}%
	{ }{}

	\newtheoremstyle{Mydefinition}%
	{1em}{1em}%
	{}{}%
	{\bfseries}{.}%
	{ }{}

	\theoremstyle{Mydefinition}
	\newtheorem{statement}{Statement}
	\newtheorem{definition}[statement]{Definition}
	\newtheorem{definitions}[statement]{Definitions}
	\newtheorem{remark}[statement]{Remark}
	\newtheorem{remarks}[statement]{Remarks}
	\newtheorem{example}[statement]{Example}
	\newtheorem{examples}[statement]{Examples}
	\newtheorem{question}[statement]{Question}
	\newtheorem{questions}[statement]{Questions}
	\newtheorem{problem}[statement]{Problem}
	\newtheorem{exercise}{Exercise}[section]
	\newtheorem*{comment*}{Comment}
	%\newtheorem{exercise}{Exercise}[subsection]

	\theoremstyle{Mytheorem}
	\newtheorem{theorem}[statement]{Theorem}
	\newtheorem{corollary}[statement]{Corollary}
	\newtheorem{corollaries}[statement]{Corollaries}
	\newtheorem{proposition}[statement]{Proposition}
	\newtheorem{lemma}[statement]{Lemma}
	\newtheorem{claim}{Claim}
	\newtheorem{claimproof}{Proof of claim}[claim]
	\newenvironment{myproof1}[1][\proofname]{%
  \proof[\textit Proof of problem #1]%
}{\endproof}

%================================================================================
%Header & footer
	\fancypagestyle{myfency}{%Plain
	\fancyhf{}
	\fancyhead[L]{}
	\fancyhead[C]{}
	\fancyhead[R]{}
	\fancyfoot[L]{}
	\fancyfoot[C]{}
	\fancyfoot[R]{\thepage}
	\renewcommand{\headrulewidth}{0.4pt}
	\renewcommand{\footrulewidth}{0pt}}

	\fancypagestyle{myfirstpage}{%Firstpage
	\fancyhf{}
	\fancyhead[L]{}
	\fancyhead[C]{}
	\fancyhead[R]{}
	\fancyfoot[L]{}
	\fancyfoot[C]{}
	\fancyfoot[R]{\thepage}
	\renewcommand{\headrulewidth}{0pt}
	\renewcommand{\footrulewidth}{0pt}}

	\pagestyle{myfency}

%================================================================================

%***************************
%*** Additional Command ****
%***************************

\DeclareMathOperator{\cl}{cl}
\DeclareMathOperator{\co}{co}
\DeclareMathOperator{\ball}{ball}
\DeclareMathOperator{\wk}{wk}
\DeclarePairedDelimiter{\ceil}{\lceil}{\rceil}
\DeclarePairedDelimiter\floor{\lfloor}{\rfloor}
\newcommand{\quotZ}[1]{\ensuremath{\mathbb{Z}/p^{#1}\mathbb{Z}}}
%================================================================================
%Document
\begin{document}
\thispagestyle{myfirstpage}
\begin{center}
	\Large{HW3}
\end{center}
박성빈, 수학과

\noindent \textbf{1}
Let $\theta_n:\mathbb{Z}\rightarrow \mathbb{Z}/p^n\mathbb{Z}$ by $1\mapsto [1]$ and $\pi_n:\mathbb{Z}_p\rightarrow \mathbb{Z}/p^n\mathbb{Z}$ by the projection onto $n$th coordinate.

Choose a point $x\in \mathbb{Z}_p$. For any $\epsilon>0$, there exists $N$ such that $p^{-N}<\epsilon$. For such $N$, there exists $X\in\mathbb{Z}$ such that $[X] = \pi_{N+1}(x)$. Since $\pi_j(x) = \theta_j(X)$ for $j\leq N$, $d(i(X), x)\leq p^{-N}<\epsilon$. Therefore, the inclusion $\mathbb{Z}$ in $\mathbb{Z}_p$ has a dense image.

\noindent \textbf{2}
$\pi_n$ is indeed surjective and the map $f:\mathbb{Z}_p\rightarrow\mathbb{Z}_p$ by $f(x) = p^n\cdot x$ is injective since $p^n\cdot(x-y) = 0$ implies $x=y$ in $\mathbb{Z}_p$. Therefore, I need to show the exactness at $\mathbb{Z}_p$.

The image of $f$ is contained in the kernel of $\pi_n$ since $f$ acts on $x\in\mathbb{Z}_p$ by shifting to right by $n$ coordinate, so $\pi_n(f(x)) = 0$. Conversely, if $y\in\mathbb{Z}_p$ is contained the kernel of $\pi_n$, then it means $\pi_i(y) = 0$ for all $i\leq n$. Let's construct $x\in \mathbb{Z}_p$ as following: for $\pi_i(y) = \sum_{j=n}^{i-1} a_j p^j$ for $i> n$ with $0\leq a_j<p$, set $\pi_{i-n}(x) = \sum_{j=0}^{i-n-1} a_{n+j} p^j$. Then $x$ is indeed in $\mathbb{Z}_p$ as $\pi_i(y) = \sum_{j=n}^{i-1} a_j p^j = p^n\left(\sum_{j=0}^{i-n-1} a_{n+j} p^j\right)$ and $\left(\sum_{j=0}^{i-n-1} a_{n+j} p^j\right)$ should satisfies the condition for $\mathbb{Z}_p$ for $i>n$. Therefore, the sequence is exact.

Since $\mathbb{Z}_p$ has a subspace topology of $\prod_{i=1}^\infty \mathbb{Z}/p^i\mathbb{Z}$, $\pi_n$ is continuous. Also, I already showed that $\mathbb{Q}_p$ is a topological ring in the previous homework, and the metric topology on $\mathbb{Z}_p$ can be viewed as a induced metric topology of $\mathbb{Q}_p$. Therefore, the multiplication $\mathbb{Z}_p\times\mathbb{Z}_p\rightarrow\mathbb{Q}_p$ is continuous. Since the image is in $\mathbb{Z}_p$, multiplication $\mathbb{Z}_p\times\mathbb{Z}_p\rightarrow\mathbb{Z}_p$ is continuous. Therefore, multiplying $p^n$ (identifying $\mathbb{Z}$ with the image in $\mathbb{Z}_p$) is continuous.

\noindent \textbf{3}

Assume $y\in p^{-n}\mathbb{Z}_p$, then $xy \in \mathbb{Z}_p$ for $x\in p^n\mathbb{Z}_p$. Therefore, $e_p(xy) = 1$ and the integral value is $\int_{p^n \mathbb{Z}_p} dx = p^{-n}$. If $y\in p^{-m}\mathbb{Z}_p\setminus p^{-n}\mathbb{Z}_p$ for some $m>n$, then there exists $z\in p^{n}\mathbb{Z}_p\setminus p^{n+1}\mathbb{Z}_p$ such that $e_p(-zy)\neq 1$. For such $z$, $1_{p^n\mathbb{Z}_p}(x) = 1_{p^n\mathbb{Z}_p}(x+z)$ and by the change of variables, $x\mapsto x+z$, we get
\begin{equation}
\begin{split}
    \int_{\mathbb{Q}_p} 1_{p^n\mathbb{Z}_p}(x)e_p(-xy)dx &= \int_{\mathbb{Q}_p} 1_{p^n\mathbb{Z}_p}(x)e_p(-xy-zy)dx \\
    &=e_p(-zy)\int_{\mathbb{Q}_p} 1_{p^n\mathbb{Z}_p}(x)e_p(-xy)dx.
\end{split}
\end{equation}
To satisfy the above equality, the integral should be $0$. Therefore, we get
\begin{equation}
    \int_{\mathbb{Q}_p} 1_{p^n\mathbb{Z}_p}(x)e_p(-xy)dx=
    \begin{cases}
        p^{-n} & y\in p^{-n}\mathbb{Z}_p\\
        0 & otherwise.
    \end{cases}
\end{equation}
%________________________________________________________________________
\end{document}

%================================================================================